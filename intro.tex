\chapter{Введение}
\indent Современные методы неизвазивного исследования мозга, такие как магнитно-резонансная томография (МРТ), в частности, диффузионно-взвешенная МРТ (дМРТ) позволяют отслеживать на макро-уровне связи между различными зонами мозга, позволяя проводить анализ сттруктурного устройства мозга. На основе дМРТ снимков возможно построить карту структурных связей между зонами мозга, представляя его таким образом в виде графа. Аналогично, с помощью функциональной МРТ (фМРТ) возможно построить граф функциональных связей между различными регионами мозга (коннектом). В этой структуре вершины графа предствляют отдельные регионы мозга, а ребра – связи между регионами. Использование коннектомов приобрело большую популярность в нейронауках из предположения, что структура связей позволяет получить понимание особенностей данной модели мозга и определить механизмы его функционирования. \\
\indent Однако диагностика психиатрических и нейродегенеративных заболеваний при помощи МРТ на сегодняшний день не является точной. Другой проблемой является малое количество данных, не позволяющее оценить обобщающую способность алгоритмов, построенных на использовании данных МРТ. \\
\indent С практической точки зрения представляет интерес задача классификации снимков МРТ, позволяющая, к примеру, автоматически определять структурные повреждения мозга, сопутствующие ранней стадии развития нейродегенеративных заболеваний. Для алгоритмов машинного обучения также является важной особенность обработки исходных снимков МРТ и построения коннектомов, поскольку на основе одного снимка возможно получить несколько разных графов в зависимости от алгоритма трактографии, используемого для преобразования. \\
В данной работе рассматривается четыре алгоритма классификации коннектомов, два из которых являются известными, например, алгоритм предложенный в \cite{dodero2015kernel}. Два других алгоритма предложены в этой работе. \\ 
\indent Ключевой особенностью этой работы является рассмотрение пространства симметричных положительно полуопределенных (СППО) и симметричных положительно определенных (СПО) матриц. Риманова геометрия предоставляет сильные инструменты для обработки структурированных данных, представленных в виде СПО матриц. СПО матрицы образуют дифференцируемое многообразие с соответствующей нелинейной структурой. Инструменты римановой геометрии позволяют находить геодезические и их длины для использования в алгоритмах, основанных на вычислении расстояний между объектами данных. В качестве другого подхода к классификации СПО матриц используется проецирование данных на касательное пространство, аппроксимирующее локальную структуру многообразия. Поскольку проекция является линейным преобразованием, спроецированные данные могут быть рассмотрены как обычные векторы.\\
\indent В последние годы инструменты римановой геометрии нашли широкое применение в анализе медицинских данных, в том числе, в задачах классификации электроэнцефалограммм (ЭЭГ) и снимков МРТ. Однако для анализа структурных снимков мозга математический аппарат многообразия СПО матриц неприменим, поскольку матрицы, представляющие структурное устройство мозга (коннектомы), не являются СПО. Эта разница является критической для алгоритмов, основанных на анализе многообразия СПО матриц.\\
В этой работе используется преобразование Лапласа матриц коннектомов, что дает в качестве данных СППО матрицы, поскольку лапласианы являются гарантированно СППО. Однако лапласианы не инвариантны относительно масштабирования соответствующих матриц смежности графов. Другое ограничение состоит в том, что для перехода в пространство СПО матриц используется регуляризация СППО матриц, однако манипуляции с диагональными элементами лапласианов затрудняют интерпретацию оператора Лапласа.\\
\indent В этой работе предложен метод, позволяющий обойти эти ограничения. Эта работа представляет фреймворк для классификации заболеваний по структурным матрицам человеческого мозга. Во-первых, данные конвертируются в нормализованные лапласианы, что позволяет избежать проблемы масштабирования. Во-вторых, задаются методы, основанные на римановой геометрии многообразия СППО матриц, что позволяет работать с лапласианами напрямую, избегая регуляризации.