\chapter{Обзор литературы}

\indent Многообразие СПО матриц было хорошо изучено в последние десятилетия и нашло широкое применение в задачах анализа медицинских данных, в первую очередь, нейрологических данных \cite{pennec2006riemannian}, \cite{fletcher2007riemannian}, таких как снимки ЭЭГ \cite{barachant2013classification}, функциональной и структурной МРТ. \\

\indent В последнее десятилетие инструменты римановой геометрии использовались в том числе для обработки DTI-изображений (diffusion tensor imaging) \cite{fletcher2007riemannian}, \cite{pennec2006riemannian}. Исследование матрицы ковариации записей ЭЭГ стали классическим подходом в построении интерфейсов «мозг-компьютер»; поскольку матрицы ковариации являются СПО матрицами, к ним возможно применить инструменты римановой геометрии. В задачах классификации ЭЭГ такие методы стали передовыми, показав значительное улучшение результатов по сравнению с существовавшими классическими методами. \cite{barachant2012multiclass}, \cite{barachant2013classification}\\

\indent Матрицы ковариации данных функциональной МРТ также привлекли внимание, учитывая быстрый рост популярности подходов, основанных на построении функциональных графов мозга (функциональных коннектомов) \cite{smith2013functional}. Функциональные коннектомы позволяют представлять функциональные связи мозга на макро-уровне. В последнее время было предложено несколько алгоритмов классификации функциональных коннектомов на основе инструментов римановой геометрии. \\
\indent Slavakis et al.\cite{slavakis2016clustering} представили новаторский подход к отслеживанию шаблонов функциональных связей мозга, варьирующихся во времени. Авторы этой работы использовали геодезические на римановском многообразии и касательные пространства к этому многообразию чтобы сгруппировать свои данных. В этой работе были получены отличные результаты на искусственно сгенерированных коннектомах. Ng et al. \cite{ng2016transport} проецировали матрицы ковариации функциональной МРТ на касательное пространство риманома многообразия с целью снизить взаимное влияние элементов ковариационных матриц; это значительно улучшило точность классификации в задаче с распознаванием четырех активностей на записях функциональной МРТ. \\
\newpage

\indent Сигналы функциональной активации между различными зонами мозга также могут быть представлены как матрицы ковариации, позволяя использовать инструменты римановой геометрии. Однако, этот подход не срабатывает для матриц структурной связности, которые содержат данные об анатомических, а не функциональных связях между регионами мозга. Элементы этих матриц представляют собой число трактов между различными регионами коры головного мозга, обычно построенные на основе диффузионно-взвешенной трактографии. Таким образом, структурные коннектомы являются неориентированными графами, а их матрицы смежности - симметричными матрицами. В противоположность матрицам функциональны связей, матрицы структурных связей не являются определенными. \\

\indent Dodero et al.\cite{dodero2015kernel} предложили метод обхода этого ограничения. Они изучили алгоритмы, основаные на геометрии Римана как для функциональных, так и для структурных коннектомов. Для структурных коннектомов они применили преобразование Лапласа на исходных данных чтобы получить симметричные положительно полуопределенные (СППО) матрицы и регуляризовали полученные матрицы через сложение их с единичной матрицей с маленьким коэффициентом. Их эксперимент на данных из базы данных снимков пациентов с расстройством аутического спектра показал точность классификации 60.76\% и 68\% для функциональных и структурных коннектомов соответственно.