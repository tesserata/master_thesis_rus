\chapter{Используемый математический аппарат}

\section{Обозначения}
\begin{itemize}
	\item $M(n)$ – пространство матриц размера $n \times n$ с действительными значениями.
	\item $ Sym(n) = \condset{\bs \in M(n)}{\bs^T \ = \ \bs}$ пространство всех симметричных матриц в $M(n)$.
	\item $ S(n) = \condset{\bs \in Sym(n)}{u^T\bs u > 0, \ \forall u \in \setRn}$ - множество всех симметричных положительно определенных (СПО) матриц размера $n \times n$.
	\item $S^+(n, p) = \condset{\bs \in Sym(n)}{u^T\bs u \geq 0, \ \forall u \in \setRn, \ rk(\bs)=p}$ – множество всех симметричных положительно полуопределенных (СППО) матриц размера $n \times n$ и ранга $p \leq n$. Это обозначение далее будет использоваться только в случае, когда $p<n$.
	\item 
	\item $Gl(n)$ – множество всех обратимых матриц размера $n \times n$ в $M(n)$.
	\item $diag(\lambda_1, \ldots, \lambda_n)$ – матрица размера $n \times n$ со значениями $\lambda_i$ на диагонали.
	\item $Tr(\cdot)$ - оператор следа матрицы
	\item $I = diag(1, \ldots, 1)$ - единичная матрица.
	\item $T_p \manM$ - касательное пространство к многообразию $\manM$ в точке $p$.
\end{itemize}

\section{Общие определения}
\begin{definition}
Норма Фробениуса для матриц: $$\norm{A}{F}^2 = Tr({\bf AA^T}) = \sum{\walls{A_{i,j}}^2}$$. 
\end{definition}
\begin{definition}
Норма $\setL_2$ для вектора ${\bf a}$: $$ \norm{{\bf a}}{2} = \sqrt{\sum_{\substack{i}}|{\bf a_i}|^2}$$
\end{definition}

\begin{definition}
Симметричная матрица $M$ называется положительно полуопределенной, если $\forall x \in \setRn$:
$$ x^T M x \geq 0 $$
$M$ положительно определена, если неравенство строгое $\forall x \neq 0$
\end{definition}

\begin{lemma}
$M$ положительно полуопределенная тогда и только тогда, когда все ее собственные значения $ \forall \lambda_i \geq 0$. Аналогично, $M$ положительно определена, тогда и только тогда, когда $\forall \lambda_i > 0$
\end{lemma}
\begin{proof}
Рассмотрим собственный базис матрицы $M=Q^T \Lambda Q$. Очевидно, что $y^T \Lambda y = \sum_{i} \lambda_i y_i^2 \geq 0 \ \forall y \in \setRn$ тогда и только тогда, когда $\lambda_i \geq 0 \forall i$. Для положительно определенных матриц аналогично.
\end{proof}

\begin{definition}
$n$-мерное топологическое многообразие (без границы) — это Хаусдорфово топологическое пространство, в котором каждая точка имеет открытую окрестность $U$, гомеоморфную открытому подмножеству $n$-мерного Евклидова пространства $\setR^n$.
\end{definition}
Далее будем обозначать $n$-мерное топологическое пространство без границы как «многообразие»
\begin{definition}
Локальная карта многообразия $\manM$ - пара $(U, \phi)$, где $\phi$ - гомеоморфизм из открытого множества $U \subset \manM$ на открытое подмножество $\setR^n$
\end{definition}

\begin{definition}
Множество карт $\set{(U_\alpha \phi_\alpha)}, \alpha \in \Alpha$ ($\Alpha$ - некоторое множество индексов) называется n-мерным $C^k$ атласом, $0 \leq k \leq \infty$ многообразия $\manM$, если:
\begin{itemize}
    \item совокупность всех $U_\alpha$ покрывает $\manM$, т.е. $\manM=\cup _{{\alpha \in A}}U_{\alpha}$.
    \item $ \forall \alpha ,\beta \in A$ таких, что $U_{\alpha }\cap U_{\beta }\neq \varnothing $, отображение:
    $$\phi _{{\alpha }}^{{\beta }}=\phi _{\beta }\circ \phi _{\alpha }^{{-1}}:\phi _{\alpha }(U_{\alpha }\cap U_{\beta })\to \phi _{\beta }(U_{\alpha }\cap U_{\beta })$$
    является гладким отображением класса $C^k$.
\end{itemize}
$ {\displaystyle \phi _{\alpha }^{\beta }}$ называется преобразованием координат точки $m$ с карты  ${\displaystyle (U_{\alpha },\phi _{\alpha })}$ в карту $(U_{\beta },\phi _{\beta })$
\end{definition}

\begin{definition}
Атлас многообразия называется $C^k$-гладким, если ${\displaystyle \phi _{\alpha }^{\beta }}$ является гладким отображением класса гладкости $C^k$
\end{definition}

\begin{definition}
Многообразие называется гладким (или дифференцируемым), если оно наделено $C^k$-гладкой структурой, задаваемой $C^k$-гладким атласом
\end{definition}

Для гладких многообразий можно ввести понятия касательного вектора, касательного пространства, длины кривой.
\section{Геометрия Риманова многообразия}

В этом разделе описаны принципы и инструменты геометрии многообразия симметричных положительно определенных матриц (Риманова многообразия.)

\subsection{Риманово многообразие}

\begin{definition}
Римановым называется многообразие $\manM$ с заданным на нем  $m \in \manM$ скалярным произведением $\left< \cdot \right>_\bc$ касательных векторов в касательном пространстве $T_p \manM$, таким, что оно гладко зависит от $p$.
\end{definition}

\begin{definition}
Риманова метрика $g$ на многообразии $\manM$ – семейство всех скалярных произведений на всех касательных пространствах
\end{definition}

\indent Далее будем рассматривать многообразие симметричных положительно определенных (СПО) матриц как пример риманова многообразия. Имеем следующие свойства: 
\begin{lemma}
Свойства пространства $S(n)$:
\begin{itemize}
	\item $\forall \ \bs \in S(n),\ det(\bs) > 0$
	\item $\forall \ \bs \in S(n),\ \bs^{-1} \in S(n)$
	\item $\forall \ (\bs_1, \bs_2) \in S(n)^2,\ \bs_1 \bs_2 \in C(n)$
\end{itemize}
\end{lemma}

\indent СПО матрицы из семейства $S(n)$ всегда диагонализуемы и имеют строго положительные действительные собственные значения. Для СПО матриц в $S(n)$ матричная экспонента $\bs$ задается через собственные значения сингулярно-векторного разложения матрицы $\bs$:
$$ \bs = {\bf U} \ diag(\sigma_1, \ldots, \sigma_n) \ {\bf U}^T, $$
где $\sigma_1 > \sigma_2 > \ldots > \sigma_n$ - собственные значения и ${\bf U}$ - матрица собственных векторов $\bs$. 
\begin{definition}
Матричная экспонента $\bs \in S(n)$: 
 $$ exp(\bs) = {\bf U} \ diag(exp(\sigma_1), \ldots, exp(\sigma_n)) \ {\bf U}^T $$
\end{definition}

\begin{definition}
Логарифм матрицы $\bs \in S(n)$:
$$ log(\bs) = {\bf U} \ diag(log(\sigma_1), \ldots, log(\sigma_n)) \ {\bf U}^T  $$ \\
\end{definition}
\indent Взятие логарифма матрицы $\bs \in S(n)$ является обратной операцией к взятию матричной экспоненты.

\begin{lemma}
Свойства пространства $S(n)$, связанные с матричными экспонентой и логарифмом:
\begin{itemize}
	\item $\forall \ \bs \in S(n),\ log(\bs) \in Sym(n)$
	\item $\forall \ \bc \in Sym(n),\ exp(\bc) \in S(n)$
	    
\end{itemize}
\end{lemma}

\subsection{Риманова метрика}
Пространство СПО матриц $S(n)$ – это риманово многообразие $\setM$, следовательно, оно дифференцируемо. Производная матрицы $\bs$ на многообразии лежит в векторном пространстве $T_{\bs}$, являющимся касательным пространством к этой точке. Касательное пространство лежит в пространстве $Sym(n)$. Многообразие и касательное пространство имеют размерность $m = n(n+1)/2$ \cite{faraut1994analysis} \\
\indent Для каждого касательного пространства определено скалярное произведение $\left< \cdot \right>_\bc$, гладко меняющееся от точки к точке. Натуральная метрика на многообразии СПО матриц определена локальным скалярным произведением: 
\begin{equation} \label{nmetr}
	 \inprod{C_1}{C_2}{\bs} = Tr(C_1\bs^{-1}C_2\bs^{-1})
\end{equation}

\indent Через скалярное произведение можем задать норму касательных векторов: \begin{equation} \label{nnorm}
	 \norm{\bc}{\bs}^2 = \inprod{\bc}{\bc}{\bs} = Tr(\bc\bs^{-1}\bc\bs^{-1})
\end{equation}

\subsection{Геодезические в римановом пространстве}
Обозначим $ \Gamma(t): \ [a,b] \rightarrow S(n) : \bs^i=\bs^i(t), i=\set{1, \ldots, n}, a \leq t \leq b$ - произвольная кривая риманова многообразия $(\manM, g)$. В любой точке этого пути касательный вектор определяется как: 
$$ v(t) =(\dot{\bs}^1(t), \ldots, \dot{\bs}^n(t) )$$
$$ |v(t)|_\bs = \sqrt{\left< v,v \right>_g\rvert_\bc} $$
Длина этой кривой определяется как: 
\begin{equation}
	L(\Gamma(t)) = \int_{a}^{b} \norm{\dot{\Gamma}(t)}{\Gamma(t)} dt = \int_{a}^{b} \sqrt{g_{ik}(\bs)\dot{\bs}^i\dot{\bs}^k} dt, \\
	g_{ik} = \left< \frac{\partial}{\partial \bs^i}, \frac{\partial}{\partial \bs^k} \right >
\end{equation}
где используется норма, определенная в предыдущем разделе формулой \eqref{nnorm}. Путь минимальной длины, соединяющий две точки многообразия, называется геодезической, и риманово расстояние между двумя точками определяется как длина этого пути. Натуральная метрика \eqref{nmetr} задает геодезическое расстояние.

\begin{equation} \label{geodd}
	\delta_{spd}(\bs_1,\bs_2) = \norm{log(\bs_1^{-1}\bs_2)}{F} = \Big{[} \sum_{i=1}^{n} log^2 \lambda_i \Big{]}^{1/2},
\end{equation} 

где $\lambda_1, \ldots, \lambda_n$ - действительные собственные значения $\bs_1^{-1}\bs_2$. Главные свойства риманова геодезического расстояния:
\begin{itemize}
	\item $\delta_{spd}(\bs_1,\bs_2) = \delta_{spd}(\bs_2, \bs_1)$
	\item $\delta_{spd}(\bs_1^{-1},\bs_2^{-1}) = \delta_{spd}(\bs_1,\bs_2)$
	\item $\delta_{spd}(\bw^T\bs_1\bw,\bw^T\bs_2\bw) = \delta_{spd}(\bs_1,\bs_2) \ \forall \ \bw \in Gl(n)$
\end{itemize}


\subsection{Экспоненциальная проекция}
Для каждой точки $\bs \in S(n)$ можно задать касательное пространство $\tau_C\mathcal{M}$, образованное множеством векторов, касательных к $\bs$. Каждый касательный вектор $\bc_i$ может рассматриваться как производная в точке $t=0$ от геодезической $\Gamma_i(t)$ между $\bs$ и экспоненциальной проекцией $\bs_i=Exp_\bs(\bc_i)$ \\
% \begin{figure}[h]
% 	\centering
% 	\includegraphics[width=0.7\textwidth]{./img/projection}
% 	\caption{Многообразие $\mathcal{M}$ и соответствующее касательное пространство $\tau_C\mathcal{M}$ в точке $\bc$}
% \end{figure} \\
\indent Экспоненциальная проекция определяется как:
\begin{equation}
	Exp_\bs(\bc_i) = \bs_i = \bs^{1/2}exp(\bs^{-1/2}\bc_i\bs^{-1/2})\bs^{1/2}
\end{equation}
\indent Обратная проекция определяется как логарифмическая проекция вида:
\begin{equation} \label{logpr}
	Log_\bs(\bs_i) = \bc_i = \bs^{1/2}log(\bs^{-1/2}\bs_i\bs^{-1/2})\bs^{1/2}
\end{equation}
\indent В терминах проекции можно ввести эквивалетное определение риманового расстояния:
$$ \delta_R(\bs_1,\bs_2) = \norm{Log_\bs(\bs_i)}{\bs} = \norm{\bc_i}{\bs} = \\ $$
$$	= \norm{upper(\bs^{-1/2}Log_\bs(\bs_i)\bs^{-1/2}}{2} = \norm{c_i}{2}, $$
где $upper(\cdot)$ - оператор, оставляющий верхнетреугольную часть симметричной матрицы и векторизующий ее, добавляя единичный вес диагональным элементам и $\sqrt{2}$ вес не-диагональным. $c_i$ здесь - $m$-мерный вектор $upper(\bs^{-1/2}Log_\bs(\bs_i)\bs^{-1/2}$ нормализованного касательного пространства.
\section{Геометрия пространства СППО матриц}
\subsection{Метрика на многообразии СППО матриц}
Можем выразить $\spn$: 
$$ \spn \cong (V_{n,p} \times S(p))/O(p) $$

Размерность $\spn$ есть $dim(V_{n,p} \times S(p)) - dim(O(p))=pn-p(p-1)/2$. \\
Если $(U, R^2) \in V_{n,p} \times S(p)$ представляет $A \in \spn$, то можно выразить касательные вектора $T_A\spn$ бесконечно малыми изменениями $(\Delta, D)$, где 
$$ \Delta = U\botB, \ B \in \setR^{(n-p) \times p} $$
$$ D = RD_0R $$
такие, что $U \bot \in V_{n, n-p}, \ U^TU\bot = 0$ и $D_0 \in Sym(p) = T_IS(p)$. Тогда метрика $\spn$ может быть задана как сумма бесконечно малых расстояний в $Gr(p,n)$ и $S(p)$:
\begin{equation}\label{spsd_metric}
     g_{(U, R^2)}((\Delta_1, D_1), (\Delta_2, D_2)) = Tr(\Delta_1^T \Delta_2) + k \ Tr(R^{-1}D_1R^{-2}D_2R^{-1}), \ k>0
\end{equation}
Следующая теорема доказывает, что введение этой метрики наделяет $\spn$ римановой структурой
\begin{theorem}
Пространство $ $\spn$ \cong (V_{n,p} \times S(p))/O(p) $, наделенное метрикой \eqref{spsd_metric} является римановым многообразием с горизонтальным пространством 
$$ \manH_{(U, R^2)} = \set{(\Delta, D):\ \Delta=U\bot B, \ B \in \setR^{(n-p) \times p},\ D=RD_0R, \ D_0 \in Sym(p)} $$
Более того, эта метрика инвариантна относительно ортогональных преобразований, масштабирования и псевдоинверсии.
\end{theorem}
Доказательство этой теоремы может быть найдено в \cite{bonnabel2009riemannian}
\newpage
\subsection{Геодезические на многообразии СППО матриц}
В этой секции рассматривается построение геодезических, соединяющих матрицы $A,B \in \spn$. \\
Пусть $V_A, V_B \in V_{n,p}$ – две матрицы, являющиеся линейными оболочками, порожденными $A, B$. Сингулярное разложение $V_B^TV_A$ порождает $O_A, O_B \in \setR^{p \times p}$ такие, что
\begin{equation} \label{OVVO}
      O_A^TV_A^TV_BO_B = diag(\sigma_1, \ldots, \sigma_p), 1 \geq \sigma_1 \geq \ldots \geq \sigma_p \geq 0
\end{equation}
$\sigma_i =  \cos \theta_i$ - косинусы углов $0 \leq \theta_1 \leq \ldots \leq \theta_p \leq \pi/2$ между двумя подпространствами.\\
Корневые вектора $ U_A=(u_1^A, \ldots, u_p^A) = V_AO_A $ и $U_B=(u_1^B, \ldots, u_p^B) = V_BO_B$ дают формулу грассмановской геодезической, соединяющей $range(A)$ и $range(B)$
\begin{equation}
     \label{grassman_geodesics}
     U(t) = U_A \cos(\Theta t)V + X \sin(\Theta t)
\end{equation}
где $\Theta = diag(\theta_1, \ldots, \Theta_p)$, а $X$ - нормализованная проекция $V$ на пространство столбцов $U\bot$, т.е. 
$X=(I-U_AU_A^T)U_BF$, где $F$ - псевдоинверсия матрицы $diag(\sin(\theta_1), \ldots, \sin(\theta_p))$. \\
Соответствующая геодезическая $R^2(t)$ в $S(p)$ должна соединять $R_A^2=U_A^TAU_A$ и $R_B^2=U_B^TBU_B$ :

\begin{equation}
     \label{p-geodesics}
     R^2(t) = R_A \exp(t\log R^{-1}_AR^2_BR^{-1}_A)R_A
\end{equation}

Таким образом геодезическая задается следующим выражением:
\begin{equation}
     \label{spsd_geodesics}
     \gamma_{A \rightarrow B}(t) = U(t)R^2(t)U^T(t)
\end{equation}
Геодезические, исходящая из любой точки, сохраняет ранг, симметричность и положительную определенность матрицы $UR^2U^T$, формируют покрытие $S^+(n,p)$, определены на $\spn$ $\forall t \in \setR$ для любого касательного вектора. \\

\subsection{Расстояние между СППО матрицами}
\begin{theorem}
Сингулярное разложение \eqref{OVVO} и геодезические кривые \eqref{grassman_geodesics} и \eqref{p-geodesics} задают кривую в $\spn$:
$$ \gamma_{A \rightarrow B}(t) = U(t)R^2(t)U^T(t) $$
со следующими свойствами:
\begin{itemize}
    \item $\gamma_{A \to B}(\cdot)$ соединяет $A$ и $B$ в $\spn$ так, что $\gamma_{A \to B}(0) = A, \ \gamma_{A \to B}(1)=B$ и $\gamma_{A \to B}(t) \in \spn \forall t \in [0, 1]$
    \item Кривая $(U(t), R^2(t))$ является горизонтальным сдвигом $\gamma_{A \to B}(t)$ и является геодезической в структурном пространстве $V_{n,p} \times S(p)$
    \item Квадратичная длина $\gamma_{A \to B}$ на римановом многообразии $(\spn, g)$ задается следующим выражением:
    \begin{equation}
        \label{spsd-distance}
         l^2(\gamma_{A \to B}) = ||\Theta||_F^2 + k ||\log R_A^{-1} R_B^2 R_A^{-1}||_F^2
    \end{equation}
    Она инвариантна относительно псевдоинверсии, ортогональных преобразований и масштабирования.
\end{itemize}
Более того, кривая $\gamma_{A \to B}$ уникальна, если $(p-1)$ угол между пространствами удовлетворяет $\theta_{p-1} \neq \pi/2$
\end{theorem}
Доказательство этой теоремы подробно рассмотрено в \cite{bonnabel2009riemannian}