\chapter{Предложенный подход}

В этой главе рассмотрим бинарную классификацию матриц структурных коннектомов. В качестве входных данных имеем множество пар данных (матриц смежности) и ответов (меток классов): $ \set{\ba_i, y_i}_{i=1}^N, \ \ba_i \in Sym(n), y_i \in \set{0,1}$, где $n$ – число вершин в каждом графе коннектома.

\section{Конвертация структурных коннектомов в СППО матрицы}
Первый шаг алгоритма состоит в конвертации структурных коннектомов в симметричные положительно полуопределенные матрицы. В работе \cite{dodero2015kernel} было предложено использовать для этого лапласианы матриц данных.

\begin{definition}
Пусть $m$ - число ребер, $n$ - чисто вершин графа. Тогда матрица смежности $\nabla = \nabla_G$ - матрица размера $m \times n$ следующего вида:
\begin{equation*} 
\nabla_{e,v} = 
\begin{cases}
1 & \quad \text{если } e=(v,w) \text{ и } v<w \\
-1 & \quad \text{если } e=(v,w) \text{ и } v>w \\
0 & \quad \text{иначе}\\
\end{cases}
\end{equation*} 
\end{definition}

\begin{lemma}
$L_G = \nabla^T \nabla$
\end{lemma}
\begin{proof}
$[\nabla^T\nabla]_{ij}=$(i-й столбец $\nabla$) $\cdot$ (j-й столбец $\nabla$)=$\sum_e\big( [\nabla]_{e,v_i} \big)\big( [\nabla]_{e,v_j} \big)$, что дает три возможных случая:
\begin{itemize}
    \item $i=j$
    $$ [\nabla^T\nabla]_{ij} = \sum_{\substack{e}}\Big( [\nabla]_{e,v_i} \Big)^2 = \sum_{\substack{e смежные с v_i}}1 = deg(i) $$
    \item $i \neq j$ и между вершинами $v_i, v_j$ нет ребра
    $$ [\nabla^T\nabla]_{ij} = \sum_{\substack{e}}\Big( [\nabla]_{e,v_i} \Big)\Big( [\nabla]_{e,v_j} \Big) = 0,$$
    поскольку ни одно ребро не смежно как минимум с одной из вершин $v_i, v_j$
    \item  $i \neq j$ и между вершинами $v_i, v_j$ есть ребро
    $$ [\nabla^T\nabla]_{ij} = \sum_{\substack{e}}\Big( [\nabla]_{e,v_i} \Big)\Big( [\nabla]_{e,v_j} \Big) = \Big( [\nabla]_{e',v_i} \Big)\Big( [\nabla]_{e',v_j} \Big)$$
\end{itemize}
\end{proof}

\begin{corollary*}
$$ x^T L_G x = ||\nabla x||^2  = \sum_{\substack{(i,j) \in E}} (x_i-x_j)^2,$$
где $E$ - множество ребер графа
\end{corollary*}

Таким образом, лапласиан является симметричной положительно полуопределенной матрицей. В предложенном подходе будем использовать нормализованные лапласианы.
Преобразование данных задается следующим образом:
$$ \bx_i = \bd_i^{-\frac{1}{2}} (\bd_i - \ba_i) \bd_i^{-\frac{1}{2}}, $$
где $\bd_i$ – диагональная матрица степеней взвешенных вершин: $ \bd_i \rvert_{k,k} = \sum_{\substack{l}} \ba_i\rvert_{k,l}$.

\section{Классификация на основе $\delta_{spsd}$}

Обобщение подхода, основанного на использовании ядерных функций, не вызывает затруднений - мы можем использовать $\delta_{spsd}$, заданное формулой \eqref{spsd-distance}, чтобы построить ядро (матрицу расстояний между полученными лапласианами). \\
Адаптация методов, основанных на проекции на касательное пространство, в случае многообразия СППО матриц, однако, гораздо сложнее, поскольку мы не можем построить касательное пространство к среднему всех СППО матриц. Чтобы обойти это ограничение, в этой работе использован алгоритм снижения размерности Isomap \cite{tenenbaum2000global} на матрицах данных  $\bx_i, i \in \set{1, \ldots, n}$, как было предложено в \cite{krivov2016dimensionality}.
Таким образом, в предложенном подходе основной упор делается на алгоритмы, основанные на ядерных функциях.